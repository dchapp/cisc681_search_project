\documentclass[12pt, conference, compsocconf]{IEEEtran}

\usepackage{amsmath, amsfonts}
\usepackage{graphicx}
\usepackage{url}
\usepackage{tikz}

%%% Package and definitions for displaying pseudocode
\usepackage[noend]{algpseudocode}
\newcommand*\Let[2]{\State #1 $\gets$ #2}
\algrenewcommand\alglinenumber[1]{
    {\sf\footnotesize\addfontfeatures{Colour=888888,Numbers=Monospaced}#1}}
\algrenewcommand\algorithmicrequire{\textbf{Precondition:}}
\algrenewcommand\algorithmicensure{\textbf{Postcondition:}}


\begin{document}
\title{CISC-481/681 Project 2: A* Search with Heuristics}

\author{\IEEEauthorblockN{Dylan Chapp, Michael Wyatt} \IEEEauthorblockA{Department of Computer and Information Sciences \\ University of Delaware - Newark, DE 19716 \\ Email: \{dchapp\}, \{mwyatt\}@udel.edu}}

\maketitle

\section{Introduction}
In this work we implement the A* search algorithm to solve the problem of directing a predator (e.g., a cat) through a maze of obstacles to multiple prey objectives (e.g., mice). 
We introduce heuristics specifically tailored to this multi-objective pathfinding problem and provide an empirical evaluation of their performance. 
Dylan

\section{Related Work}
Mike

\subsection{Problem Characterization}
We are presented with the problem of providing a cat with a shortest path that allows it to navigate to and consume $m$ mice on an $n \times n$ board of tiles. 
However, some of these tiles are obstacles that the cat cannot traverse, which makes computing shortest paths nontrivial.
We employ the A* search algorithm to find, for a fixed starting tile and a fixed goal tile containing a mouse, a shortest path for the cat. 
In turn, we use heuristics for executing repeated applications of single-objective A* search to find a shortest path to all of the mice. 

\section{The A* Algorithm}
A* is an informed search algorithm used for finding the shortest path from a single node in a graph--the start--to some other node in the graph--the goal. 
Specifically, A* constructs a tree whose root is the start node and one of which's leaves is the goal node. 

The A* algorithm bears resemblance to uninformed search algorithms such as breadth-first search in terms of its general structure, but differs in that it maintains a priority queue of nodes to expand upon rather than a FIFO/LIFO queue. 
The priority queue is ordered according to a heuristic that accounts for both the known cost to reach a node from the starting node and the estimated cost to reach the goal node from that node. 

In fact, A* was developed as an improved upon Dijkstra's algorithm, a similiar search algorithm that too maintains a priority queue of nodes, but orders them only with respect to the known cost to reach them from the starting node. 
Below, we provide pseudocode for A* highlighting the difference.

\begin{algorithm}
    \caption{A* Search}
    \label{alg-astar}
    \begin{algorithmic}[1]
        \Statex
        \Function{A*}{start, goal} 
            \Let{explored}{\{\}}
            \Let{frontier}{\{\}}
            \While{}
            
            \EndWhile
        \State \Return{0}
        \EndFunction
    \end{algorithmic}
\end{algorithm}
            
            

\subsection{Single Objective A* Heuristics}
Mke

\subsection{Multi Objective A* Heuristics}
Mike

\section{Implementation}
Dylan 

\section{Evaluation}

\subsection{Runtime Performance}
Dylan

\subsection{Optimality}
Mike

\section{Future Work}
Mike

\section{Conclusion}
Dylan

\bibliographystyle{IEEEtran}
\bibliography{bibliography}

\end{document}
